\documentclass[a4paper,10pt]{article}

\usepackage{amsmath}
\usepackage[spanish]{babel}

\setlength{\parskip}{3pt plus 2pt}
\setlength{\parindent}{20pt}
\setlength{\oddsidemargin}{0.5cm}
\setlength{\evensidemargin}{0.5cm}
\setlength{\marginparsep}{0.75cm}
\setlength{\marginparwidth}{2.5cm}
\setlength{\marginparpush}{1.0cm}
\setlength{\textwidth}{150mm}

\begin{document}

\begin{center}
Sumatoria finita cuadr\'atica \\
Facundo Mat\'ias Carreiro \\
\today
\end{center}

\section{Motivaci\'on}
Cuando estaba resolviendo un ejercicio de Probabilidad y Estad\'istica me
top\'e con un n\'umero que necesitaba calcular, \'este era:
\begin{displaymath}
\sum_{k=0}^{5} k^2 = 1^2 + 2^2 + 3^2 + 4^2 + 5^2 = 55
\end{displaymath}
En este caso todo era facil porque simplemente deb\'ia sumar los $5$ primeros
n\'umeros naturales pero eso no ser\'ia tan f\'acil si necesitara la suma
hasta $50$ por ejemplo. Lo que me hizo pensar: habr\'a una f\'ormula general? \\
Durante varios d\'ias trat\'e de resolverlo haciendo cambios de indice y
tratando de buscar alg\'un tipo de analog\'ia con respecto a la muy conocida
f\'ormula de Gauss para sumar los primeros $n$ n\'umeros:
\begin{equation}
\sum_{k=0}^{n} k = \frac{n(n+1)}{2}
\end{equation}
pero los cambios de variables cuadr\'aticas no se pod\'ian aplicar en la
sumatoria (o al menos no logr\'e descubrir como).

\section{B\'usqueda de patrones}
Para encontrar una f\'ormula general para el caso cuadr\'atico me pareci\'o
buena idea hacerme una tablita con los primeros casos llamando $g(n)$ al
resultado de aplicar la f\'ormula de Gauss al n\'umero n y $f(n)$ a la
sumatoria cuadr\'atica que intentaba resolver:

\begin{table}[h]
\begin{center}
\begin{tabular}{|c|c|c|c|}
\hline
$n$ & $g(n)$ & $n^2$ & $f(n)$ \\
\hline
1 & 1 & 1 & 1 \\
\hline
2 & 3 & 4 & 5 \\
\hline
3 & 6 & 9 & 14 \\
\hline
4 & 10 & 16 & 30 \\
\hline
\end{tabular}
\end{center}
\end{table}
Despu\'es de mirar la tabla por un tiempo y probar varias relaciones que
funcionaban para algunos casos pero para otros no me pareci\'o notar que se
cumpl\'ia
\begin{displaymath}
\sum_{k=0}^{2} k^2 = 2.g(2) - \sum_{k=0}^{2-1} g(k) =
2.g(2) - g(0) - g(1) =
6 - 0 - 1 = 5
\end{displaymath}
Lo que era a\'un m\'as interesante es que se cumpl\'ia para todos los casos
que ten\'ia escritos, m\'as generalmente
\begin{equation}
\sum_{k=0}^{n} k^2 = n.g(n) - \sum_{k=0}^{n-1} g(k)
\end{equation}
Que reemplazando por las funciones originales queda
\begin{equation}
\sum_{k=0}^{n} k^2 = n.\left(\frac{n.(n+1)}{2}\right) -
\sum_{k=0}^{n-1} \frac{(k+1).k}{2}
\end{equation}
En este punto me frustr\'e un poco porque vi que lo mismo que quer\'ia
probar estaba practicamente igual dentro de la soluci\'on pero despu\'es
record\'e un truco que usabamos para resolver integrales y empec\'e a
reducir la expresi\'on
\begin{displaymath}
\sum_{k=0}^{n} k^2 = n.\left(\frac{n.(n+1)}{2}\right) -
\sum_{k=0}^{n-1} \frac{(k+1).k}{2} \Longleftrightarrow
\end{displaymath}
\begin{displaymath}
\sum_{k=0}^{n} k^2 = n.\left(\frac{n^2 + n}{2}\right) -
\sum_{k=0}^{n-1} \frac{k^2 + k}{2} \Longleftrightarrow
\end{displaymath}
\begin{displaymath}
\sum_{k=0}^{n} k^2 = \frac{n^3 + n^2}{2} -
\frac{1}{2} \left( \sum_{k=0}^{n-1} k^2 + k \right) \Longleftrightarrow
\end{displaymath}
\begin{displaymath}
\sum_{k=0}^{n} k^2 = \frac{n^3 + n^2}{2} -
\frac{1}{2} \sum_{k=0}^{n-1} k^2 - 
\frac{1}{2} \sum_{k=0}^{n-1} k
\end{displaymath}
Ahora puedo reducir la sumatoria de $0$ hasta $n-1$ usando la f\'ormula de
Gauss
\begin{displaymath}
\sum_{k=0}^{n} k^2 = \frac{n^3 + n^2}{2} -
\frac{1}{2} \frac{(n-1).n}{2} -
\frac{1}{2} \sum_{k=0}^{n-1} k^2 \Longleftrightarrow
\end{displaymath}
\begin{displaymath}
\sum_{k=0}^{n} k^2 = \frac{n^3 + n^2}{2} -
\frac{n^2 - n}{4} -
\frac{1}{2} \sum_{k=0}^{n-1} k^2 \Longleftrightarrow
\end{displaymath}
\begin{displaymath}
\sum_{k=0}^{n} k^2 = \frac{2n^3 + 2n^2 - n^2 + n}{4} -
\frac{1}{2} \sum_{k=0}^{n-1} k^2 \Longleftrightarrow
\end{displaymath}
\begin{displaymath}
\sum_{k=0}^{n} k^2 = \frac{2n^3 + n^2 + n}{4} -
\frac{1}{2} \sum_{k=0}^{n-1} k^2
\end{displaymath}
Ahora tengo que sumar y restar un t\'ermino m\'as de la sumatoria de la
derecha del igual para que quede algo igual a lo de la izquierda del igual
\begin{displaymath}
\sum_{k=0}^{n} k^2 = \frac{2n^3 + n^2 + n}{4} +
\frac{1}{2} n^2 - \frac{1}{2} \sum_{k=0}^{n} k^2
\end{displaymath}
Pasando la sumatoria para el otro lado ahora me queda
\begin{displaymath}
\frac{3}{2} \sum_{k=0}^{n} k^2 = \frac{2n^3 + n^2 + n}{4} +
\frac{1}{2} n^2 \Longleftrightarrow
\end{displaymath}
\begin{equation}
\sum_{k=0}^{n} k^2 = \frac{2n^3 + 3n^2 + n}{6}
\end{equation}

\section{Demostraci\'on}
Para probar que la f\'ormula vale para todos los naturales (y el $0$) us\'e
el m\'etodo de inducci\'on:
\begin{equation}
P(0) \longrightarrow \sum_{k=0}^{n} k^2 = \frac{2n^3 + 3n^2 +n}{6}
\Longleftrightarrow 0 = 0
\end{equation}
Ahora usando como Hip\'otesis Inductiva que
\begin{displaymath}
\sum_{k=0}^{n} k^2 = \frac{2n^3 + 3n^2 +n}{6}
\end{displaymath}
Debo probar que
\begin{displaymath}
\sum_{k=0}^{n+1} k^2 = \frac{2(n+1)^3 + 3(n+1)^2 + (n+1)}{6} \Longleftrightarrow
\end{displaymath}
\begin{displaymath}
6 \sum_{k=0}^{n} k^2 + 6(n+1)^2 = 2(n+1)^3 + 3(n+1)^2 + (n+1) \Longleftrightarrow
\end{displaymath}
\begin{displaymath}
6 \sum_{k=0}^{n} k^2 + 6(n+1)^2 = 2(n+1)^2(n+1) + 3(n^2 + 2n + 1) + (n + 1) \Longleftrightarrow
\end{displaymath}
\begin{displaymath}
6 \sum_{k=0}^{n} k^2 + 6(n+1)^2 = 2(n^2 + 2n + 1)(n+1) + 3(n^2 + 2n + 1) + (n + 1) \Longleftrightarrow
\end{displaymath}
\begin{displaymath}
6 \sum_{k=0}^{n} k^2 + 6(n+1)^2 = 2(n^3 + n^2 + 2n^2 + 2n + n + 1) + 3(n^2 + 2n + 1) + (n + 1) \Longleftrightarrow
\end{displaymath}
\begin{displaymath}
6 \sum_{k=0}^{n} k^2 + 6(n+1)^2 = 2(n^3 + 3n^2 + 3n + 1) + 3(n^2 + 2n + 1) + (n + 1) \Longleftrightarrow
\end{displaymath}
\begin{displaymath}
6 \sum_{k=0}^{n} k^2 + 6(n+1)^2 = 2n^3 + 6n^2 + 6n + 2 + 3n^2 + 6n + 3 + n + 1 \Longleftrightarrow
\end{displaymath}
\begin{displaymath}
6 \sum_{k=0}^{n} k^2 + 6(n+1)^2 = 2n^3 + 9n^2 + 13n + 6 \Longleftrightarrow
\end{displaymath}
\begin{displaymath}
6 \sum_{k=0}^{n} k^2 + 6(n^2 + 2n + 1) = 2n^3 + 9n^2 + 13n + 6 \Longleftrightarrow
\end{displaymath}
\begin{displaymath}
6 \sum_{k=0}^{n} k^2 = 2n^3 + 9n^2 + 13n + 6 - 6n^2 - 12n - 6\Longleftrightarrow
\end{displaymath}
\begin{displaymath}
\sum_{k=0}^{n} k^2 = \frac{2n^3 + 3n^2 + n}{6}
\end{displaymath}
Y esto es cierto por Hip\'otesis Inductiva, queda probado que la f\'ormula vale
para todo n\'umero natural o cero.

\end{document}
